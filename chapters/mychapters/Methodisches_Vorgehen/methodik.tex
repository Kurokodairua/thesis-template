\chapter{Methodisches Vorgehen}
Im Zuge dieser Arbeit wird Literatur zu den Themen Turing Tests, CAPTCHA, Honeypots und UX gesichtet, 
um auf Basis des mithilfe dieser Quellen beschriebenen Grundlagenwissens fortlaufend Schlussfolgerungen treffen zu können. 

Nachfolgend wird geprüft, welche Alternativen es zur Bewertung von CAPTCHA-Techniken gibt, bevor eine solche selbst entwickelt wird. 
Dies geschieht in Form einer Bewertungsmatrix, aus der sich mit gegebener Gewichtung eine Metrik beziehungsweise ein „Score“ für verschiedene CAPTCHA-Techniken berechnen lässt. 
Zusätzlich werden Alternativen zu CAPTCHA betrachtet und dahingehend evaluiert, ob die entwickelte Matrix auch anwendbar ist.

%warum  werden kategorien gewählt

Zu besseren Veranschaulichung werden die verschiedenen CAPTCHA-Arten mithilfe eines Bildbearbeitungsprogramms nachgestellt. 
Dabei werden unterschiedliche Darstellungsarten unabhängig von Produkten am Markt dargestellt.

Auf Basis dieser Matrix werden diese Techniken für verschiedene Ausgangsszenarien, 
aus denen sich erwähnte Gewichtungen ergeben, bewertet, und es werden Empfehlungen ausgesprochen.

Anschließend werden die gefundenen Ergebnisse auf ihre Plausibilität hin geprüft und es wird ein Fazit gezogen, 
in wie weit die entwickelte Matrix und die daraus berechenbaren Scores anwendbar sind.
  
