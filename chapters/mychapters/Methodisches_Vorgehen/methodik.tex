\chapter{Methodisches Vorgehen}
Im Zuge dieser Arbeit wird Literatur zu den Themen Turing Tests, CAPTCHA, Alternativen zu klassischen CAPTCHAs und UX gesichtet.
Auf Basis des mithilfe dieser Quellen beschriebenen Grundlagenwissens werden fortlaufend Schlussfolgerungen getroffen. 

Nachfolgend wird geprüft, welche Alternativen es zur Bewertung von CAPTCHA-Techniken gibt, bevor eine solche Bewertungstechnik selbst entwickelt wird. 
Dies geschieht in Form einer Bewertungsmatrix, aus der sich mit gegebener Gewichtung eine Metrik beziehungsweise ein „Score“ für verschiedene CAPTCHA-Techniken berechnen lässt. 
Zusätzlich werden Alternativen zu CAPTCHA betrachtet und dahingehend evaluiert, ob die entwickelte Matrix auch anwendbar ist.

Die verschiedenen in Rahmen dieser Arbeit untersuchten CAPTCHA-Arten werden beschrieben und mit Screenshots veranschaulicht.

Auf Basis dieser Matrix werden diese Techniken für verschiedene Ausgangsszenarien, 
aus denen sich erwähnte Gewichtungen ergeben, bewertet, und es werden Empfehlungen ausgesprochen,
welche CAPTCHA-Arten in diesen Szenarien besonders gut geeignet sind.

Anschließend werden die gefundenen Ergebnisse auf ihre Plausibilität hin geprüft und es wird ein Fazit gezogen, 
in wie weit die entwickelte Matrix und die daraus berechenbaren Scores anwendbar sind.
  
