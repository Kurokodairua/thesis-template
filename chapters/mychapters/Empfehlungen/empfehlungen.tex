\chapter{Empfehlungen}
\label{ch:empfehlung}

Basierend auf den im vorherigen Kapitel aufgestellten Bewertungen werden für verschiedene fiktive Szenarien Scores berechnet 
und mithilfe dieser Empfehlungen hinsichtlich der Wahl des CAPTCHA ausgesprochen. 
Dies soll einerseits als Beispiel zur Anwendung dienen, 
andererseits auch eine erste Orientierung bei der Auswahl der zu prüfenden Techniken geben. 

\subsubsection*{Beispiel 1}
Als erstes Szenario soll eine Bank dienen, die für ihr Online Banking mit CAPTCHAs arbeitet.
Sicherheit ist hier besonders wichtig, 
da eine Störung des Systems fatal für Kund*innen und die Bank selbst sein könnte.
Außerdem sollte möglichst allen Kund*innen der Zugriff möglich sein, 
weshalb Accessability ebenfalls eine große Rolle spielt.
Aus diesem Grund kann in Sachen Bedienfreundlichkeit 
und auch bei der technischen Umsetzbarkeit zu Gunsten der Ausfallsicherheit zurückgesteckt werden.
Daraus ergeben sich folgende Gewichtungen:

\begin{table}[h!]
    \caption{Gewichtung für Beispiel 1}
    \begin{center}
        \begin{tabular}{l|c}
            Kategorie                       & Gewichtung \\\hline
            Bedienfreundlichkeit            & 15\%         \\
            Accessibility                   & 30\%        \\
            Technische Umsetzbarkeit        & 15\%         \\
            Sicherheit                      & 40\%         
        \end{tabular}
    \end{center}
\end{table}

In \autoref{tabellen1} sind die ausführlichen Matrizen für Beispiel 1 zu finden. 
Aus diesen Tabellen lassen sich folgende Bewertungen entnehmen: 

\begin{table}[h!]
    \caption{Bewertungen für Beispiel 1}
    \begin{center}
        \begin{tabular}{l|c}
            CAPTCHA-Art                       & Gesamtnote \\\hline
            textbasiert            &  5.55-7.85       \\
            bildbasiert                   &  6.3-8.4      \\
            audiobasiert        & 5.75-5.9         \\
            invisible*                      & 9.05         \\
            Honeypots       & 8.35\\
            \multicolumn{2}{l}{\footnotesize * Invisible steht hier für jegliche CAPTCHAs} \\
            \multicolumn{2}{l}{\footnotesize \space \space mit minimalem User-Input}
        \end{tabular}
    \end{center}
\end{table}

Aus den vorliegenden Ergebnissen lässt sich erkennen, dass Invisible CAPTCHAs die höchste Gesamtnote erreichen konnten.
Insbesondere bei der Accessibility können sie überzeugen, 
da sie durch ihr Arbeiten im Hintergrund keine Interaktion verlangen und somit von allen Nutzer*innen problemlos benutzt werden können.
Außerdem sind CAPTCHAs mit wenig User-Input relativ sicher, da fast ausschließlich Metadaten betrachtet werden und das genaue Verfahren unbekannt ist.
Dadurch ist es schwerer, diese Daten für Bots nachzustellen.

Ähnlich verhält es sich im Best Case bei bildbasierten CAPTCHAs, insbesondere wenn hier Mausbewegungen und Ähnliches mit in Betracht gezogen werden.



- Honeypots

- wieso audio nicht so gut ist

Textbasierte CAPTCHAs erzielen im Worst Case den niedrigen Score.
Dies liegt unter anderem an der niedrigen Bewertung im Bereich der Sicherheit, 
da es hier viele Methoden gibt, textbasierte CAPTCHAs durch Algorithmen erkennen und lösen zu lassen.
Auch im Bereich der Accessibility muss darauf geachtet werden, dass es ausreichend Alternativen für Menschen gibt,
die Probleme mit visuellen CAPTCHAs haben könnten.

\dots

Eine Kombination verschiedener CAPTCHA-Arten, wie es beispielsweise bei reCAPTCHA oder hCaptcha der Fall ist, 
kann eventuelle Mängel bei einzelnen Techniken beheben.

\subsubsection*{Beispiel 2}
Das zweite Szenario soll ein kleines Forum sein, in dem ein CAPTCHA ausgefüllt wird, bevor ein Beitrag veröffentlicht wird.
Hierbei ist besonders auf die Bedienfreundlichkeit zu achten, damit Nutzer*innen sich mit möglichst wenig Irritation austauschen können.
Außerdem soll die gewählte Technologie möglichst einfach technisch umzusetzen sein, um Kosten und Ressourcenverbrauch gering halten zu können.
Accessibility ist zwar auch von Relevanz, jedoch nicht so stark wie zuvor genannte Kategorien.
Die Sicherheit ist aufgrund der geringen Anzahl an abgegebenen Beiträgen vergleichsweise unwichtig.

\begin{table}[h!]
    \caption{Gewichtung für Beispiel 2}
    \begin{center}
        \begin{tabular}{l|c}
            Kategorie                       & Gewichtung \\\hline
            Bedienfreundlichkeit            & 35\%         \\
            Accessibility                   & 20\%        \\
            Technische Umsetzbarkeit        & 35\%         \\
            Sicherheit                      & 10\%         
        \end{tabular}
    \end{center}
\end{table}

Analog zu Beispiel 1 sind die genauen Bewertungsmatrizen in \autoref{tabellen2} hinterlegt.
Es ergeben sich folgende Gesamtnoten:

\begin{table}[h!]
    \caption{Bewertungen für Beispiel 2}
    \begin{center}
        \begin{tabular}{l|c}
            CAPTCHA-Art                       & Gesamtnote \\\hline
            textbasiert            & 7.45-9.15        \\
            bildbasiert                   & 7.2-9.6       \\
            audiobasiert        & 7.25-7.4         \\
            invisible*                      & 9.45         \\
            Honeypots & 9.15 \\
           \multicolumn{2}{l}{\footnotesize * Invisible steht hier für jegliche CAPTCHAs} \\
           \multicolumn{2}{l}{\footnotesize   \enspace mit minimalem User-Input}
        \end{tabular}
    \end{center}
\end{table}