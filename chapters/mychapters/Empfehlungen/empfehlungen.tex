\chapter{Empfehlungen}
\label{ch:empfehlung}

Basierend auf den im vorherigen Kapitel aufgestellten Bewertungen werden für verschiedene fiktive Szenarien Scores berechnet 
und mithilfe dieser Empfehlungen hinsichtlich der Wahl des CAPTCHA ausgesprochen. 
Dies soll einerseits als Beispiel zur Anwendung dienen, 
andererseits auch eine erste Orientierung bei der Auswahl der zu prüfenden Techniken geben. 

\subsubsection*{Beispiel 1}
Als erstes Szenario soll eine Bank dienen, die für ihr Online Banking mit CAPTCHAs arbeitet.
Sicherheit ist hier besonders wichtig, 
da eine Störung des Systems fatal für Kund*innen und die Bank selbst sein könnte.
Außerdem sollte möglichst allen Kund*innen der Zugriff möglich sein, 
weshalb Accessability ebenfalls eine große Rolle spielt.
Aus diesem Grund kann in Sachen Bedienfreundlichkeit 
und auch bei der technischen Umsetzbarkeit zu Gunsten der Ausfallsicherheit zurückgesteckt werden.
Daraus ergeben sich folgende Gewichtungen:

\begin{table}[h!]
    \caption{Gewichtung für Beispiel 1}
    \begin{center}
        \begin{tabular}{l|c}
            Kategorie                       & Gewichtung \\\hline
            Bedienfreundlichkeit            & 15\%         \\
            Accessibility                   & 30\%        \\
            Technische Umsetzbarkeit        & 15\%         \\
            Sicherheit                      & 40\%         
        \end{tabular}
    \end{center}
\end{table}

In \autoref{tabellen1} sind die ausführlichen Matrizen für Beispiel 1 zu finden. 
Aus diesen Tabellen lassen sich folgende Bewertungen entnehmen: 

\begin{table}[h!]
    \caption{Bewertungen für Beispiel 1}
    \begin{center}
        \begin{tabular}{l|c}
            CAPTCHA-Art                       & Gesamtnote \\\hline
            textbasiert            &  5.55-7.85       \\
            bildbasiert                   &  6.3-8.4      \\
            audiobasiert        & 5.75-5.9         \\
            invisible*                      & 9.05         \\
            Honeypots       & 8.35\\
            \multicolumn{2}{l}{\footnotesize * Invisible steht hier für jegliche CAPTCHAs} \\
            \multicolumn{2}{l}{\footnotesize \space \space mit minimalem User-Input}
        \end{tabular}
    \end{center}
\end{table}

Aus den vorliegenden Ergebnissen lässt sich erkennen, dass Invisible CAPTCHAs die höchste Gesamtnote erreichen konnten.
Insbesondere bei der Accessibility können sie überzeugen, 
da sie durch ihr Arbeiten im Hintergrund keine Interaktion verlangen und somit von allen Nutzer*innen problemlos benutzt werden können.
Außerdem sind CAPTCHAs mit wenig User-Input relativ sicher, da fast ausschließlich Metadaten betrachtet werden und das genaue Verfahren unbekannt ist.
Dadurch ist es schwerer, diese Daten für Bots nachzustellen.

Ähnlich verhält es sich im Best Case bei bildbasierten CAPTCHAs, besonders wenn hier Mausbewegungen und Ähnliches mit in Betracht gezogen werden.
Zwar ist hier die direkte Mitarbeit der Nutzer*innen nötig, jedoch handelt es sich im besten Fall um wenige Klicks. 
Es kann jedoch auch vorkommen, dass der bildbasierte CAPTCHA sehr kompliziert oder unverständlich ist und mehrere Versuche notwendig sind,
weshalb sie nicht die beste Wahl sind.
Außerdem ist es durch verschiedene Algorithmen zur Bildbearbeitung und den Einsatz von künstlicher Intelligenz möglich,
eine Vielzahl von Objekten korrekt zu erkennen.

Auch Honeypots erzielen einen relativ hohen Score.
Dies liegt daran, dass sie ebenfalls mit sehr wenig Nutzerinteraktion verbunden sind
und dementsprechend auch wenig Probleme im Bereich der Accessibility bereiten.
Hier besteht jedoch das Problem, dass ein einmal erkannter Honeypot bei jedem weiteren Versuch umgangen werden kann.
Damit wird das System schlagartig wesentlich anfälliger für Spam-Angriffe, da der Sicherheitsmechanismus nicht mehr zuschlagen kann.
Aus diesem Grund sind Honeypots nicht zu empfehlen, wenn es sich um Systeme handelt, die eine sehr hohe Sicherheit verlangen.

Audiobasierte CAPTCHAs konnten keine hohe Gesamtnote erreichen.
Dies hängt unter anderem mit ihrer geringen Bewertung im Bereich Accessibility zusammen,
da sie zwar als Alternative zu visuellen CAPTCHAs sehr gut funktionieren, jedoch als alleinige Lösung einige Probleme mit sich bringen.
So schließen sie nicht nur gehörlose Menschen gänzlich aus, sondern können auch hörende Menschen den CAPTCHA eventuell nicht absolvieren,
da sie sich in Situationen befinden, in denen sie keine Töne abspielen wollen oder (technisch) können.

Textbasierte CAPTCHAs erzielen im Worst Case den niedrigsten Score, doch auch im Best Case sind sie auf dem vorletzten Platz.
Dies liegt unter anderem an der niedrigen Bewertung im Bereich der Sicherheit, 
da es hier viele Methoden gibt, textbasierte CAPTCHAs durch Algorithmen erkennen und lösen zu lassen.
Auch im Bereich der Accessibility muss darauf geachtet werden, dass es ausreichend Alternativen für Menschen gibt,
die Probleme mit visuellen CAPTCHAs haben könnten.

Eine Kombination verschiedener CAPTCHA-Arten, wie es beispielsweise bei reCAPTCHA oder hCaptcha der Fall ist, 
kann eventuelle Mängel bei einzelnen Techniken beheben.
Oftmals werden bildbasierte CAPTCHAs als Backup für Invisible CAPTCHA verwendet, sollten die Metadaten keine genauen Schlüsse zulassen.
Das Anbieten von audiobasierten CAPTCHAs als weitere Option zu einem visuellen CAPTCHA ist ebenfalls zu empfehlen.
So kann eine maximale Accessibility erreicht werden.

Im Bereich der Sicherheit ist vor allem darauf zu achten, stets die neuste Version des gewählten CAPTCHAs zu nutzen 
und sich über eventuelle Sicherheitslücken zu informieren.

\subsubsection*{Beispiel 2}
Das zweite Szenario soll ein kleines Forum sein, in dem ein CAPTCHA ausgefüllt wird, bevor ein Beitrag veröffentlicht wird.
Hierbei ist besonders auf die Bedienfreundlichkeit zu achten, damit Nutzer*innen sich mit möglichst wenig Irritation austauschen können.
Außerdem soll die gewählte Technologie möglichst einfach technisch umzusetzen sein, um Kosten und Ressourcenverbrauch gering halten zu können.
Accessibility ist zwar auch von Relevanz, jedoch nicht so stark wie zuvor genannte Kategorien.
Die Sicherheit ist aufgrund der geringen Anzahl an abgegebenen Beiträgen vergleichsweise unwichtig.

\begin{table}[h!]
    \caption{Gewichtung für Beispiel 2}
    \begin{center}
        \begin{tabular}{l|c}
            Kategorie                       & Gewichtung \\\hline
            Bedienfreundlichkeit            & 35\%         \\
            Accessibility                   & 20\%        \\
            Technische Umsetzbarkeit        & 35\%         \\
            Sicherheit                      & 10\%         
        \end{tabular}
    \end{center}
\end{table}

Analog zu Beispiel 1 sind die genauen Bewertungsmatrizen in \autoref{tabellen2} hinterlegt.
Es ergeben sich folgende Gesamtnoten:

\begin{table}[h!]
    \caption{Bewertungen für Beispiel 2}
    \begin{center}
        \begin{tabular}{l|c}
            CAPTCHA-Art                       & Gesamtnote \\\hline
            textbasiert            & 7.45-9.15        \\
            bildbasiert                   & 7.2-9.6       \\
            audiobasiert        & 7.25-7.4         \\
            invisible*                      & 9.45         \\
            Honeypots & 9.15 \\
           \multicolumn{2}{l}{\footnotesize * Invisible steht hier für jegliche CAPTCHAs} \\
           \multicolumn{2}{l}{\footnotesize   \enspace mit minimalem User-Input}
        \end{tabular}
    \end{center}
\end{table}

Auch hier erreichen CAPTCHAs mit geringem User-Input eine sehr gute Bewertung.
Die höchste Gesamtnote erreichen im Best Case jedoch bildbasierte CAPTCHAs.

Es ist zu beachten, dass es sich hier nur um eine Diskrepanz von 0.3 Notenpunkten handelt. 
Sie entsteht dadurch, dass bei bildbasierten CAPTCHAs weniger Interaktion nötig sein kann als bei einem invisible CAPTCHA,
welcher aufgrund eines zu niedrigen Scores zusätzliche Verifikation, 
wie das Lösen eines bild- oder audiobasierten CAPTCHAs, benötigt.
Im Bereich der technischen Umsetzbarkeit können beide Methoden voll überzeugen. 
Die Nutzung von APIs ermöglicht eine schnelle Implementation.

Textbasierte CAPTCHAs erzielen im Best Case ebenfalls eine hohe Punktzahl. 
Dies liegt daran, dass besonders der Punkt Sicherheit keine große Rolle spielt
und die niedrige Bewertung die Gesamtnote deshalb nicht so stark beeinflusst wie es bei Beispiel 1 der Fall ist. 

Durch die große Spannweite in den Kategorien Bedienfreundlichkeit und Accessibility, 
liegen die Bewertungen des Worst Case und Best Case bei textbasierten CAPTCHAs weit auseinander.
Hier ist es also stark von der Einzellösung abhängig, ob diese wirklich eine gute Wahl für diesen Anwendungsfall darstellen würde.

Ähnlich verhält es sich auch bei bildbasierten CAPTCHAs. 

Auch Honeypots bilden eine gute Alternative. 
Da die Sicherheit in diesem Beispiel nur gering gewertet wird, ist die Problematik, dass sobald ein Honeypot erkannt wurde,
kein Schutz mehr besteht, nicht allzu relevant. 
In Sachen Bedienfreundlichkeit und Accessability verhalten sich Honeypots analog zu invisible CAPTCHAs,
jedoch ohne eventuell benötigte zusätzliche Verifikation.

Audiobasierte CAPTCHAs können trotz ihrer relativ hohen Bedienfreundlichkeit und technischer Umsetzbarkeit keinen hohen Score erzielen.
Das liegt daran, dass sie eine sehr niedrige Bewertung in den Kategorien Accessability und Sicherheit haben,
was sich trotz ihrer vergleichsweise niedrigen Bewertung stark auf die Gesamtnote auswirkt.

Gewählt werden sollte hier ein CAPTCHA, das möglichst einfach auszufüllen ist, wie insivible CAPTCHAs oder bildbasierte CAPTCHAs. 
Honeypots können aufgrund ihrer hohen Bedienfreundlichkeit ebenfalls genutzt werden, 
sind technisch jedoch etwas schwieriger umzusetzen als andere Methoden.