\chapter{Bewertung verschiedener Methoden}

Nachfolgend werden die verschiedenen Arten von CAPTCHA sowie einige Alternativen, wie Honeypots, Anti-Spam-Plugins, 
Multi-Faktor-Authentifikation und biometrische Verfahren auf Basis der zuvor entwickelten Matrix bewertet. 

\section{Textbasierte CAPTCHA}

Durch ihren simplen Aufbau sind textbasierte CAPTCHA relativ einfach zu verstehen.
Insbesondere einfache mathematische Aufgaben lassen sich auch gut audiobasiert umsetzen, 
um die Accessibility des CAPTCHAs zu gewährleisten.
Bei optisch verzerrten Wörtern besteht jedoch die Möglichkeit, dass es zu Missverständnissen kommen kann.
So können beispielsweise ein kleines ``d'' und ein ``cl'' unter Umständen verwechselt werden. (Vgl. \cite[p.3]{usabilityofcaptchas})

Hollow: A main feature of hollow CAPTCHAs is to use
contour lines to form connected characters with the aim of
improving security and usability simultaneously, as the
connected characters are hard to segment, but are easily seen
by humans. 
Unfortunately, this mechanism is not as secure as
people expected, research by Gao [17] used a generic method
to combine segmentation with recognition to break a series of
really hollow CAPTCHAs with success rates ranging from
36% to 89%. 

CCT and overlapping: crowing characters together
(CCT) and overlapping try to make segmentation more
difficult by squeezing characters together. However, it may
reduce user friendliness. For instance, reference [13] broke the
Google CAPTCHA and reCAPTCHA [14] with success rates
of 46.75% and 33%, respectively. A novel method was also
presented in [15] to attack CCT-based CAPTCHAs, achieving
success rates from 27.1% to 53.2%.

Noise background: The noise background mechanism
hides the position of the characters. Regrettably, Google’s
reCAPTCHA, which uses Street View images, is broken by a
method imitating the probability of a sequence proposed in
[21]. In addition, Gao [19] used some image processing
techniques iteratively to break PayPal’s CAPTCHA.

Two-layer structure: A two-layer structure is a vertical
combination of two single-layers CAPTCHA. The most
critical issue in breaking this mechanism is segmentation,
which cannot be solved by common segmentation methods.
Gao [18] presented a novel two-dimensional segmentation
approach to separate a CAPTCHA image along both vertical
and horizontal directions and achieved a success rate of
44.6%.
76

Multifonts, Rotation, Waving: These three methods are
designed to increase the diversity of each character, thereby
increasing the number of features needed to identify each class
by machine.

Large character set: A large character set makes the
solution space much larger than that of traditional text
CAPTCHAs. It usually chose Chinese, Japanese, etc. 
\cite{surveyofresearch}

\begin{table}[h!]
    \caption{Beispielhafte Bewertungsmatrix}
    \begin{center}
        \begin{tabular}{l|c}
            Kategorie                       & Bewertung \\\hline
            Aufwand                         & 7         \\
            Accessibility                   & 10        \\
            Technische Umsetzbarkeit        & 3         \\
            Sicherheit                      & 9         
        \end{tabular}
    \end{center}
\end{table}

\section{Bildbasierte CAPTCHA}

\begin{table}[h!]
    \caption{Beispielhafte Bewertungsmatrix}
    \begin{center}
        \begin{tabular}{l|c}
            Kategorie                       & Bewertung \\\hline
            Aufwand                         & 7         \\
            Accessibility                   & 10        \\
            Technische Umsetzbarkeit        & 3         \\
            Sicherheit                      & 9         
        \end{tabular}
    \end{center}
\end{table}

\section{Audiobasierte CAPTCHA}

“Buster” ist ein Browser Addon, das reCAPTCHA audio challenges per speech recognition löst.


A. Audio-based CAPTCHA
This CAPTCHA is usually considered an alternative to a
visual CAPTCHA in the case of visually impaired users [44].
Users in most audio-based CAPTCHAs play the role of
listeners, and they are required to complete the specified
challenge based on what they have heard. A spoken
CAPTCHA system was introduced in [45]. This system
converts a selected word into speech using a Text-To-Speech
(TTS) system, then plays the sound clip to users and asks them
to say the word. In 2012, the SoundsRight audio CAPTCHA
(Fig. 6(a)) provided in [48] asks users to identify a specific
sound, such as the sound of a bell or a piano. This work has
increased the success rates in audio Captchas from less than
50% to over 90% for blind users. Meutzner et al. presented a
new type of audio CAPTCHA [50] that uses additional
nonsense speech sounds that are confusing for speech
recognizers, while being less critical for human listeners. In
2016, they also proposed a nonspeech audio CAPTCHA [61],
which is entirely based on the classification of sound events
mixed into an environmental scene. Moreover, the HuMan
CAPTCHA designed in [60] asks users to answer the
presented questions by combining ambient noise and common
sense knowledge. There is another type of audio-based
CAPTCHA in which users play the role of speakers and are
required to pronounce rather than simply listen. For instance,
Gao et al. [47] proposed a new sound-based CAPTCHA (Fig.
6(b)) that exploits the differences between a human voice a
synthetic voice. A user is required to read out a given sentence
rather than listening an audio file.
Fig. 6. Examples of audio-based CAPTCHAs
Attack and defense always go together. A two-stage attack
for the listener model can always obtain a good attack result.
In detail, the audio-based CAPTCHA is segmented into
several regions regarding the location of each spoken word
first. Then, the regions are recognized by automatic speech
recognition programs. Tam et al. achieved success rates of up
to 71% (Google Audio CAPTCHA, reCAPTCHA Audio
CAPTCHA, Digg CAPTCHA)[44]. Bursztein et al. achieved
success rates of 45%, 49% and 83% on the CAPTCHAs of
Yahoo, Microsoft and eBay, respectively[49]. Some
78
Authorized licensed use limited to: Fachhochschule FH Darmstadt. Downloaded on July 29,2022 at 08:34:41 UTC from IEEE Xplore. Restrictions apply.
researchers have even proposed that most of the digit-based
audio CAPTCHAs are successfully broken with success rates
between 50%-90% [44][52][54].

\begin{table}[h!]
    \caption{Beispielhafte Bewertungsmatrix}
    \begin{center}
        \begin{tabular}{l|c}
            Kategorie                       & Bewertung \\\hline
            Aufwand                         & 7         \\
            Accessibility                   & 10        \\
            Technische Umsetzbarkeit        & 3         \\
            Sicherheit                      & 9         
        \end{tabular}
    \end{center}
\end{table}

\section{Logik-CAPTCHA}

\begin{table}[h!]
    \caption{Beispielhafte Bewertungsmatrix}
    \begin{center}
        \begin{tabular}{l|c}
            Kategorie                       & Bewertung \\\hline
            Aufwand                         & 7         \\
            Accessibility                   & 10        \\
            Technische Umsetzbarkeit        & 3         \\
            Sicherheit                      & 9         
        \end{tabular}
    \end{center}
\end{table}

\section{Gamification}

\begin{table}[h!]
    \caption{Beispielhafte Bewertungsmatrix}
    \begin{center}
        \begin{tabular}{l|c}
            Kategorie                       & Bewertung \\\hline
            Aufwand                         & 7         \\
            Accessibility                   & 10        \\
            Technische Umsetzbarkeit        & 3         \\
            Sicherheit                      & 9         
        \end{tabular}
    \end{center}
\end{table}

\section{Alternativen zu CAPTCHA}

\subsection{Honeypot}

\begin{table}[h!]
    \caption{Beispielhafte Bewertungsmatrix}
    \begin{center}
        \begin{tabular}{l|c}
            Kategorie                       & Bewertung \\\hline
            Aufwand                         & 7         \\
            Accessibility                   & 10        \\
            Technische Umsetzbarkeit        & 3         \\
            Sicherheit                      & 9         
        \end{tabular}
    \end{center}
\end{table}

\subsection{Anti Spam Plugins}

\begin{table}[h!]
    \caption{Beispielhafte Bewertungsmatrix}
    \begin{center}
        \begin{tabular}{l|c}
            Kategorie                       & Bewertung \\\hline
            Aufwand                         & 7         \\
            Accessibility                   & 10        \\
            Technische Umsetzbarkeit        & 3         \\
            Sicherheit                      & 9         
        \end{tabular}
    \end{center}
\end{table}

\subsection{Multi-Faktor Authentifikation}

\begin{table}[h!]
    \caption{Beispielhafte Bewertungsmatrix}
    \begin{center}
        \begin{tabular}{l|c}
            Kategorie                       & Bewertung \\\hline
            Aufwand                         & 7         \\
            Accessibility                   & 10        \\
            Technische Umsetzbarkeit        & 3         \\
            Sicherheit                      & 9         
        \end{tabular}
    \end{center}
\end{table}

\subsection{Biometrie}

\begin{table}[h!]
    \caption{Beispielhafte Bewertungsmatrix}
    \begin{center}
        \begin{tabular}{l|c}
            Kategorie                       & Bewertung \\\hline
            Aufwand                         & 7         \\
            Accessibility                   & 10        \\
            Technische Umsetzbarkeit        & 3         \\
            Sicherheit                      & 9         
        \end{tabular}
    \end{center}
\end{table}