\chapter{Bewertung verschiedener Methoden}

Nachfolgend werden die verschiedenen Arten von CAPTCHA sowie einige Alternativen, wie Honeypots, Anti-Spam-Plugins, 
Multi-Faktor-Authentifikation und biometrische Verfahren auf Basis der zuvor entwickelten Matrix bewertet. 

\section{Textbasierte CAPTCHA}

Durch ihren simplen Aufbau sind textbasierte CAPTCHA relativ einfach zu verstehen.
Insbesondere einfache mathematische Aufgaben lassen sich auch gut audiobasiert umsetzen, 
um die Accessibility des CAPTCHAs zu gewährleisten.
Bei optisch verzerrten Wörtern besteht jedoch die Möglichkeit, dass es zu Missverständnissen kommen kann.
So können beispielsweise ein kleines ``d'' und ein ``cl'' unter Umständen verwechselt werden. (Vgl. \cite[p.3]{usabilityofcaptchas})

Hollow: A main feature of hollow CAPTCHAs is to use
contour lines to form connected characters with the aim of
improving security and usability simultaneously, as the
connected characters are hard to segment, but are easily seen
by humans. 
Unfortunately, this mechanism is not as secure as
people expected, research by Gao [17] used a generic method
to combine segmentation with recognition to break a series of
really hollow CAPTCHAs with success rates ranging from
36% to 89%. 

CCT and overlapping: crowing characters together
(CCT) and overlapping try to make segmentation more
difficult by squeezing characters together. However, it may
reduce user friendliness. For instance, reference [13] broke the
Google CAPTCHA and reCAPTCHA [14] with success rates
of 46.75% and 33%, respectively. A novel method was also
presented in [15] to attack CCT-based CAPTCHAs, achieving
success rates from 27.1% to 53.2%.

Noise background: The noise background mechanism
hides the position of the characters. Regrettably, Google’s
reCAPTCHA, which uses Street View images, is broken by a
method imitating the probability of a sequence proposed in
[21]. In addition, Gao [19] used some image processing
techniques iteratively to break PayPal’s CAPTCHA.

Two-layer structure: A two-layer structure is a vertical
combination of two single-layers CAPTCHA. The most
critical issue in breaking this mechanism is segmentation,
which cannot be solved by common segmentation methods.
Gao [18] presented a novel two-dimensional segmentation
approach to separate a CAPTCHA image along both vertical
and horizontal directions and achieved a success rate of
44.6%.
76

Multifonts, Rotation, Waving: These three methods are
designed to increase the diversity of each character, thereby
increasing the number of features needed to identify each class
by machine.

Large character set: A large character set makes the
solution space much larger than that of traditional text
CAPTCHAs. It usually chose Chinese, Japanese, etc. 
\cite{surveyofresearch}

\begin{table}[h!]
    \caption{Beispielhafte Bewertungsmatrix}
    \begin{center}
        \begin{tabular}{l|c}
            Kategorie                       & Bewertung \\\hline
            Aufwand                         & 7         \\
            Accessibility                   & 10        \\
            Technische Umsetzbarkeit        & 3         \\
            Sicherheit                      & 9         
        \end{tabular}
    \end{center}
\end{table}

\section{Bildbasierte CAPTCHA}

\begin{table}[h!]
    \caption{Beispielhafte Bewertungsmatrix}
    \begin{center}
        \begin{tabular}{l|c}
            Kategorie                       & Bewertung \\\hline
            Aufwand                         & 7         \\
            Accessibility                   & 10        \\
            Technische Umsetzbarkeit        & 3         \\
            Sicherheit                      & 9         
        \end{tabular}
    \end{center}
\end{table}

\section{Audiobasierte CAPTCHA}

“Buster” ist ein Browser Addon, das reCAPTCHA audio challenges per speech recognition löst.

\begin{table}[h!]
    \caption{Beispielhafte Bewertungsmatrix}
    \begin{center}
        \begin{tabular}{l|c}
            Kategorie                       & Bewertung \\\hline
            Aufwand                         & 7         \\
            Accessibility                   & 10        \\
            Technische Umsetzbarkeit        & 3         \\
            Sicherheit                      & 9         
        \end{tabular}
    \end{center}
\end{table}

\section{Logik-CAPTCHA}

\begin{table}[h!]
    \caption{Beispielhafte Bewertungsmatrix}
    \begin{center}
        \begin{tabular}{l|c}
            Kategorie                       & Bewertung \\\hline
            Aufwand                         & 7         \\
            Accessibility                   & 10        \\
            Technische Umsetzbarkeit        & 3         \\
            Sicherheit                      & 9         
        \end{tabular}
    \end{center}
\end{table}

\section{Gamification}

\begin{table}[h!]
    \caption{Beispielhafte Bewertungsmatrix}
    \begin{center}
        \begin{tabular}{l|c}
            Kategorie                       & Bewertung \\\hline
            Aufwand                         & 7         \\
            Accessibility                   & 10        \\
            Technische Umsetzbarkeit        & 3         \\
            Sicherheit                      & 9         
        \end{tabular}
    \end{center}
\end{table}

\section{Alternativen zu CAPTCHA}

\subsection{Honeypot}

\begin{table}[h!]
    \caption{Beispielhafte Bewertungsmatrix}
    \begin{center}
        \begin{tabular}{l|c}
            Kategorie                       & Bewertung \\\hline
            Aufwand                         & 7         \\
            Accessibility                   & 10        \\
            Technische Umsetzbarkeit        & 3         \\
            Sicherheit                      & 9         
        \end{tabular}
    \end{center}
\end{table}

\subsection{Anti Spam Plugins}

\begin{table}[h!]
    \caption{Beispielhafte Bewertungsmatrix}
    \begin{center}
        \begin{tabular}{l|c}
            Kategorie                       & Bewertung \\\hline
            Aufwand                         & 7         \\
            Accessibility                   & 10        \\
            Technische Umsetzbarkeit        & 3         \\
            Sicherheit                      & 9         
        \end{tabular}
    \end{center}
\end{table}

\subsection{Multi-Faktor Authentifikation}

\begin{table}[h!]
    \caption{Beispielhafte Bewertungsmatrix}
    \begin{center}
        \begin{tabular}{l|c}
            Kategorie                       & Bewertung \\\hline
            Aufwand                         & 7         \\
            Accessibility                   & 10        \\
            Technische Umsetzbarkeit        & 3         \\
            Sicherheit                      & 9         
        \end{tabular}
    \end{center}
\end{table}

\subsection{Biometrie}

\begin{table}[h!]
    \caption{Beispielhafte Bewertungsmatrix}
    \begin{center}
        \begin{tabular}{l|c}
            Kategorie                       & Bewertung \\\hline
            Aufwand                         & 7         \\
            Accessibility                   & 10        \\
            Technische Umsetzbarkeit        & 3         \\
            Sicherheit                      & 9         
        \end{tabular}
    \end{center}
\end{table}