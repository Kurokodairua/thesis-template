\chapter{Bewertung verschiedener Methoden}

Nachfolgend werden die verschiedenen Arten von CAPTCHA sowie einige Alternativen, wie Honeypots, Anti-Spam-Plugins, 
Multi-Faktor-Authentifikation und biometrische Verfahren auf Basis der zuvor entwickelten Matrix bewertet. 

\section{Textbasierte CAPTCHA}

Durch ihren simplen Aufbau sind textbasierte CAPTCHA relativ einfach zu verstehen.
Insbesondere einfache mathematische Aufgaben lassen sich auch gut audiobasiert umsetzen, 
um die Accessibility des CAPTCHAs zu gewährleisten.
Bei optisch verzerrten Wörtern besteht jedoch die Möglichkeit, dass es zu Missverständnissen kommen kann.
So können ein kleines ``d'' und ein ``cl'' unter Umständen verwechselt werden.

\section{Bildbasierte CAPTCHA}

\section{Audiobasierte CAPTCHA}

“Buster” ist ein Browser Addon, das reCAPTCHA audio challenges per speech recognition löst.

\section{Logik-CAPTCHA}

\section{Gamification}

\section{Alternativen zu CAPTCHA}

\subsection{Honeypot}

\subsection{Anti Spam Plugins}

\subsection{Multi-Faktor Authentifikation}

\subsection{Biometrie}