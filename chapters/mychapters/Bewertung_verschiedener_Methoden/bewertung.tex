\chapter{Bewertung verschiedener Methoden}

Nachfolgend werden die verschiedenen Arten von CAPTCHA sowie einige Alternativen, wie Honeypots, Anti-Spam-Plugins, 
Multi-Faktor-Authentifikation und biometrische Verfahren auf Basis der zuvor entwickelten Matrix bewertet. 

\section{Textbasierte CAPTCHA}

Durch ihren simplen Aufbau sind textbasierte CAPTCHA relativ einfach zu verstehen.
Es sind wenige Clicks nötig, um einen solchen CAPTCHA auszufüllen. 

Accessability ist ambivalent zu betrachten.
Insbesondere einfache mathematische Aufgaben lassen sich auch gut audiobasiert umsetzen, 
um die Accessibility des CAPTCHAs zu gewährleisten.
Textbasierte CAPTCHA allein sind jedoch für sehbehinderte Menschen nicht nutzbar.
Bei optisch verzerrten Wörtern besteht jedoch die Möglichkeit, dass es zu Missverständnissen kommen kann.
So können beispielsweise ein kleines ``d'' und ein ``cl'' unter Umständen auch von Menschen verwechselt werden. (Vgl. \cite[p.3]{usabilityofcaptchas})

\citeauthor{surveyofresearch} schreiben in \cite[p.xx]{surveyofresearch}, 
dass Hollow CAPTCHA durch eine Kombination von Segmentierung und Recognition %TODO
mit einer Erfolgsrate von 36 bis 89 Prozent durch Bots gelöst werden konnten. (Vgl. \cite{gao}) %TODO

Ähnlich verhält es sich bei Überlappungen und CCT (``crowing characters together''):
Hier konnte bei verschiedenen Technologien mithilfe von unterschiedlichen Methoden
eine Erfolgsquote von 27,1 bis 53,2 Prozent erreicht werden. \cite[p.xx]{surveyofresearch} %Quellen aus Text noch mit aufnehmen

Auch bei ``noise backgrounds'' und ``two-layer structures'' konnten solche Ergebnisse erzielt werden. 
(Vgl. \cite[p.xx]{surveyofresearch} \cite{gao})

%Multifonts, Rotation, Waving, Large character set \cite{surveyofresearch}

\begin{table}[h!]
    \caption{Beispielhafte Bewertungsmatrix}
    \begin{center}
        \begin{tabular}{l|c}
            Kategorie                       & Bewertung \\\hline
            Aufwand                         & 7-9         \\
            Accessibility                   & 5        \\
            Technische Umsetzbarkeit        & 10         \\
            Sicherheit                      & 3-5         
        \end{tabular}
    \end{center}
\end{table}

\section{Bildbasierte CAPTCHA}

%Golle [28] proposed an SVM (Support Vector Machine)
%classifier to distinguish the images of cats and dogs in Asirra
%with an 82.7% success rate. In [36], Gao’s team utilized
%OpenCV functions to detect faces in the FR-CAPTCHA, and
%four features were extracted from the faces to find the most
%probable pair. Reference [39] leveraged deep learning
%technologies to break an image reCAPTCHA and the
%Facebook CAPTCHA with success rates of 70.78% and
%83.5%, respectively.
%Selection-based CAPTCHAs are simple and convenient
%for users to operate. However, they have gradually become
%vulnerable due to the development of deep learning.

\begin{table}[h!]
    \caption{Beispielhafte Bewertungsmatrix}
    \begin{center}
        \begin{tabular}{l|c}
            Kategorie                       & Bewertung \\\hline
            Aufwand                         & 7         \\
            Accessibility                   & 10        \\
            Technische Umsetzbarkeit        & 3         \\
            Sicherheit                      & 9         
        \end{tabular}
    \end{center}
\end{table}

%B. Video-based CAPTCHA
%In video-based CAPTCHAs, users are provided a video
%file, and they should choose one option that best matches the
%video. Motion CAPTCHAs (Fig. 7(a)) proposed in [46] asks
%users to select the sentence that describes the motion of the
%person in the video. Rao et al. [55] proposed a video
%CAPTCHA (Fig. 7(b)) based on advertisement recognition.
%Both CAPTCHAs need users to select from the options%
%provided. That is, these schemes can be broken by random
%guessing. Contrastly, reference [42] presented a video-based
%CAPTCHA (Fig. 7(c)), which asks users to type three words
%that best describe a video.
%Fig. 7. Examples of video-based CAPTCHAs
%Due to recent advances, bandwidth and video analysis
%technology no longer limit the development of video-based
%CAPTCHAs. In 2015, Sano et al. first used HMM-based
%(hidden Markov model-based) attacks to successfully attack a
%video-based reCAPTCHA from Google with a 31.75%
%success rate [65]. 

%In 2008, Richard Chow et al.[23] first proposed the clickbased CAPTCHA. It requires users to click characters in a
%complex background according to a short hint, as shown in
%Fig. 3. This CAPTCHA simplifies the user's operation,
%shortens the passing time and minimizes users’ frustration.
%Fig. 3. Examples of click-based CAPTCHAs
%In general, click-based CAPTCHAs have two defense
%mechanisms: anti-detection and anti-recognition. It is no
%longer a difficult task to recognize characters correctly with
%the development of machine learning. Therefore, almost all
%security mechanisms focus on preventing attackers from
%correctly detecting characters. As shown in Fig. 3(c), the
%CAPTCHA uses the style transfer technique [24] to embed
%characters into the background to achieve the effect of hiding
%the characters.
%Recently, a novel click-based CAPTCHA named VTT was
%proposed by Tencent (see Fig. 4). For a computer, it is
%difficult to understand the semantic information and analyze
%the image content as well as humans. In this regard, it seems a
%good design. However, state-of-the-art research on visual
%reasoning, such as [58] and [8], may break this scheme in the
%near future.

%The drag-based CAPTCHA judges whether the user is a
%human through the mouse’s track, speed and response time.
%The operation of a drag-based CAPTCHA is simple. Some
%drag-based CAPTCHAs are shown in Fig. 5.
%Fig. 5. Examples of drag-based CAPTCHAs
%The What’s Up CAPTCHA, proposed by Google [37], is
%the first drag-based CAPTCHA. Users need to identify the
%upright orientations of randomly rotated images and adjust
%them to the correct position. Setting an image to an upright
%orientation is easy for humans, whereas it is difficult for bots.
%It is worth noting that images used in CAPTCHA must be
%manually filtered from of samples that do not contain clear
%directional information. After this development, GeeTest
%proposed the first version of a slider CAPTCHA. Users must
%drag a slider along a line to the specified position
%continuously. Inspired by this, the VAPTCHA appeared. It
%asks users to draw a trajectory with a mouse according to an
%arrow trajectory embedded in the background.
%In fact, early drag-based CAPTCHAs judged the
%legitimacy of users only by measuring the speed of their
%operation. Therefore, it can be easily imitated. The later dragbased CAPTCHAs usually incorporate some user background
%data analysis. At present, this seems to be the future
%development trend for CAPTCHAs

\section{Audiobasierte CAPTCHA}

“Buster” ist ein Browser Addon, das reCAPTCHA audio challenges per speech recognition löst.
%
%A. Audio-based CAPTCHA
%This CAPTCHA is usually considered an alternative to a
%visual CAPTCHA in the case of visually impaired users [44].
%Users in most audio-based CAPTCHAs play the role of
%listeners, and they are required to complete the specified
%challenge based on what they have heard. A spoken
%CAPTCHA system was introduced in [45]. This system
%converts a selected word into speech using a Text-To-Speech
%(TTS) system, then plays the sound clip to users and asks them
%to say the word. In 2012, the SoundsRight audio CAPTCHA
%(Fig. 6(a)) provided in [48] asks users to identify a specific
%sound, such as the sound of a bell or a piano. This work has
%increased the success rates in audio Captchas from less than
%50% to over 90% for blind users. Meutzner et al. presented a
%new type of audio CAPTCHA [50] that uses additional
%nonsense speech sounds that are confusing for speech
%recognizers, while being less critical for human listeners. In
%2016, they also proposed a nonspeech audio CAPTCHA [61],
%which is entirely based on the classification of sound events
%mixed into an environmental scene. Moreover, the HuMan
%CAPTCHA designed in [60] asks users to answer the
%presented questions by combining ambient noise and common
%sense knowledge. There is another type of audio-based
%CAPTCHA in which users play the role of speakers and are
%required to pronounce rather than simply listen. For instance,
%Gao et al. [47] proposed a new sound-based CAPTCHA (Fig.
%6(b)) that exploits the differences between a human voice a
%synthetic voice. A user is required to read out a given sentence
%rather than listening an audio file.
%Fig. 6. Examples of audio-based CAPTCHAs
%Attack and defense always go together. A two-stage attack
%for the listener model can always obtain a good attack result.
%In detail, the audio-based CAPTCHA is segmented into
%several regions regarding the location of each spoken word
%first. Then, the regions are recognized by automatic speech
%recognition programs. Tam et al. achieved success rates of up
%to 71% (Google Audio CAPTCHA, reCAPTCHA Audio
%CAPTCHA, Digg CAPTCHA)[44]. Bursztein et al. achieved
%success rates of 45%, 49% and 83% on the CAPTCHAs of
%Yahoo, Microsoft and eBay, respectively[49]. Some
%78
%Authorized licensed use limited to: Fachhochschule FH Darmstadt. Downloaded on July 29,2022 at 08:34:41 UTC from IEEE Xplore. Restrictions apply.
%researchers have even proposed that most of the digit-based
%audio CAPTCHAs are successfully broken with success rates
%between 50%-90% [44][52][54].


\begin{table}[h!]
    \caption{Beispielhafte Bewertungsmatrix}
    \begin{center}
        \begin{tabular}{l|c}
            Kategorie                       & Bewertung \\\hline
            Aufwand                         & 7         \\
            Accessibility                   & 10        \\
            Technische Umsetzbarkeit        & 3         \\
            Sicherheit                      & 9         
        \end{tabular}
    \end{center}
\end{table}

\section{Logik-CAPTCHA}

\begin{table}[h!]
    \caption{Beispielhafte Bewertungsmatrix}
    \begin{center}
        \begin{tabular}{l|c}
            Kategorie                       & Bewertung \\\hline
            Aufwand                         & 7         \\
            Accessibility                   & 10        \\
            Technische Umsetzbarkeit        & 3         \\
            Sicherheit                      & 9         
        \end{tabular}
    \end{center}
\end{table}

\section{Gamification}

\begin{table}[h!]
    \caption{Beispielhafte Bewertungsmatrix}
    \begin{center}
        \begin{tabular}{l|c}
            Kategorie                       & Bewertung \\\hline
            Aufwand                         & 7         \\
            Accessibility                   & 10        \\
            Technische Umsetzbarkeit        & 3         \\
            Sicherheit                      & 9         
        \end{tabular}
    \end{center}
\end{table}

\section{Alternativen zu CAPTCHA}

\subsection{Honeypot}

\begin{table}[h!]
    \caption{Beispielhafte Bewertungsmatrix}
    \begin{center}
        \begin{tabular}{l|c}
            Kategorie                       & Bewertung \\\hline
            Aufwand                         & 7         \\
            Accessibility                   & 10        \\
            Technische Umsetzbarkeit        & 3         \\
            Sicherheit                      & 9         
        \end{tabular}
    \end{center}
\end{table}

\subsection{Anti Spam Plugins}

\begin{table}[h!]
    \caption{Beispielhafte Bewertungsmatrix}
    \begin{center}
        \begin{tabular}{l|c}
            Kategorie                       & Bewertung \\\hline
            Aufwand                         & 7         \\
            Accessibility                   & 10        \\
            Technische Umsetzbarkeit        & 3         \\
            Sicherheit                      & 9         
        \end{tabular}
    \end{center}
\end{table}

\subsection{Multi-Faktor Authentifikation}

\begin{table}[h!]
    \caption{Beispielhafte Bewertungsmatrix}
    \begin{center}
        \begin{tabular}{l|c}
            Kategorie                       & Bewertung \\\hline
            Aufwand                         & 7         \\
            Accessibility                   & 10        \\
            Technische Umsetzbarkeit        & 3         \\
            Sicherheit                      & 9         
        \end{tabular}
    \end{center}
\end{table}

\subsection{Biometrie}

\begin{table}[h!]
    \caption{Beispielhafte Bewertungsmatrix}
    \begin{center}
        \begin{tabular}{l|c}
            Kategorie                       & Bewertung \\\hline
            Aufwand                         & 7         \\
            Accessibility                   & 10        \\
            Technische Umsetzbarkeit        & 3         \\
            Sicherheit                      & 9         
        \end{tabular}
    \end{center}
\end{table}