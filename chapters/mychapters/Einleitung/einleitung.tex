\chapter{Einleitung}
\label{ch:intro}

Für die Unterscheidung von Menschen und Maschine bei der Betreibung von Webseiten werden verschiedene Methoden genutzt.
Eine davon sind sogenannte CAPTCHA - ``completely automated turing tests to tell computers and humans apart''.
Besonders wenn es um das Ausfüllen von Anmelde- oder Bestellformularen geht, sind diese häufig anzutreffen.

Das Ausfüllen von CAPTCHAs unterbricht oftmals die normale Nutzung von Webseiten und kann als störend empfunden werden.

Ihre Nutzung ist notwendig, um übermäßige Belastungen von Webseiten und den Missbrauch fremder Daten einzudämmen oder ganz zu verhindern.

Um die Nutzererfahrung so angenehm wie möglich gestalten zu können,
muss sorgfältig abgewägt werden, welcher CAPTCHA eingesetzt werden sollte. 

Aus diesem Grund soll eine Möglichkeit geschaffen werden, um CAPTCHAs und ihre Alternativen bewerten zu können.
%\section{Motivation}
%\label{sec:intro:motivation}


%
% Section: Ziele
%
\section{Ziel der Arbeit}
\label{sec:intro:goal}

Ziel der Arbeit ist es, eine Bewertungsmatrix für CAPTCHA-Technologien zu bieten, welche bei der Auswahl einer geeigneten Methode unterstützen soll.
Es sind folgende Leitfragen zu klären:

Kann auf Basis von zuvor gewählten Kategorien eine aussagekräftige Matrix zur
Bewertung verschiedener CAPTCHA-Methoden und ihrer Alternativen entwickeln werden?

Kann auf Basis dieser Matrix und einer gegebenen Gewichtung
ein vergleichbarer Score für bestimmte Einsatzgebiete berechnet werden?

%
% Section: Struktur der Arbeit
%
\section{Gliederung}
\label{sec:intro:structure}
In der folgenden Arbeit werden verschiedene CAPTCHA-Methoden, sowie einige Alternativen zu klassischen CAPTCHAs,
hinsichtlich ihres Einflusses auf die Nutzung von Webseiten betrachtet.

Hierfür wird eine Bewertungsmatrix entwickelt, mithilfe derer die unterschiedlichen Techniken basierend auf
ihrer Bedienfreundlichkeit beim Ausfüllen, ihrer Accessibility, der technischen Umsetzbarkeit $($Implementation, Wartung$)$ und ihrer Sicherheit beurteilt werden können.

Auf Basis dieser Bewertung und einer für den Anwendungsfall individuell gewählter Gewichtung wird anschließend eine Score-Berechnung 
an verschiedenen Beispielen vorgestellt, sodass Evaluierungen leicht vergleichbar gemacht werden können.