\chapter{Einleitung}
\label{ch:intro}

Für die Unterscheidung von Menschen und Maschine bei der Betreibung von Webseiten werden verschiedene Methoden genutzt.
Eine davon sind sogenannte CAPTCHA - ``completely automated turing tests to tell computers and humans apart''.
Besonders wenn es um das Ausfüllen von Anmelde- oder Bestellformularen geht, sind diese häufig anzutreffen.

Das Ausfüllen von CAPTCHAs unterbricht die normale Nutzung von Webseiten und kann als störend empfunden werden.

Doch ihre Nutzung ist notwendig, um übermäßige Belastungen von Webseiten und den Missbrauch fremder Daten einzudämmen oder ganz zu verhindern.


%\section{Motivation}
%\label{sec:intro:motivation}


%
% Section: Ziele
%
\section{Ziel der Arbeit}
\label{sec:intro:goal}
Es ist die Frage zu klären, ob man auf Basis von zuvor gewählten Kategorien eine aussagekräftige Matrix zur
Bewertung verschiedener CAPTCHA-Methoden und ihrer Alternativen entwickeln kann, und ob auf Basis dieser Matrix und einer gegebenen Gewichtung
ein vergleichbarer Score für bestimmte Einsatzgebiete berechnet werden kann.

Ziel der Arbeit ist es, eine Bewertungsmatrix für CAPTCHA-Technologien zu bieten, welche bei der Auswahl einer geeigneten Methode unterstützen soll.
%
% Section: Struktur der Arbeit
%
\section{Gliederung}
\label{sec:intro:structure}
In der folgenden Arbeit werden verschiedene CAPTCHA-Methoden, sowie einige Alternativen zu klassischen CAPTCHAs,
hinsichtlich ihres Einflusses auf die Nutzung von Webseiten betrachtet.

Hierfür wird eine Bewertungsmatrix entwickelt, mithilfe derer die unterschiedlichen Techniken basierend auf
ihrem Aufwand beim Ausfüllen, ihrer Accessibility, der technischen Umsetzbarkeit $($Implementation, Wartung$)$ und ihrer Sicherheit beurteilt werden können.

Auf Basis dieser Bewertung und einer individuellen Gewichtung wird anschließend eine Score-Berechnung an verschiedenen Beispielen vorgestellt,
sodass Evaluierungen leicht vergleichbar gemacht werden können.