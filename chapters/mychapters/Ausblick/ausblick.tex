\chapter{Ausblick}

Künstliche Intellligenzen bewegen sich immer näher darauf hin, Turing Tests bestehen zu können.
Bereits 2014 behauptete Kevin Warwick, dass der Chatbot ``Eugene Goostman'' den Turing Test bestanden hätte,
da dieser von 33\% seiner Chatpartner bei einem Event davon überzeugen konnte, ein Mensch zu sein. \cite{eugene}

Im Zuge dieser stetigen Entwicklung müssen auch CAPTCHAs weiterentwickelt werden,
um nicht obsolet zu werden. 

Dies hat zur Folge, dass CAPTCHAs sich in zwei Richtungen bewegen können:

Einerseits müssen klassische visuelle oder audiobasierte CAPTCHAs stetig komplizierter gestaltet werden,
um vorhandenen Algorithmen zur Bild- und Spracherkennung entgegenwirken zu können.

Andererseits kann auch vermehrt auf Metadaten und Ähnliches zurückgegriffen werden, was eventuelle Datenschutzeinschränkungen zur Folge hat.

Als Lösung für dieses Problem werden schon heute die verschiedenen CAPTCHA-Arten miteinander kombiniert.
Ist sich beispielsweise reCAPTCHA oder hCaptcha nicht sicher, ob es sich bei der Nutzer*in um einen Menschen handelt,
müssen weitere CAPTCHAs ausgefüllt werden.

Neben der Nutzung von vorhandenen Produkten auf dem Markt können bei Bedarf auch eigene CAPTCHAs entwickelt werden.
Dies gestaltet sich jedoch aufgrund der bereits genannten Entwicklungen stetig komplizierter.

Die Grenze, wann etwas noch als CAPTCHA bezeichnet werden kann, vist schwierig zu ziehen.
So handelt es sich bei invisible CAPTCHAs auch nicht um klassische Tests mit Nutzereingaben.

Eventuell gibt es deshalb in Zukunft eine neue Bezeichnung inklusive neuer Definition, welche CAPTCHAs ablösen könnte.