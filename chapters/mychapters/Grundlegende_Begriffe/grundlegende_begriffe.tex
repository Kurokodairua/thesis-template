\chapter{Grundlegende Begriffe}
\label{ch:basics}
In folgendem Kapitel werden die Grundlagen erläutert, welche für spätere Kapitel benötigt werden. 
Es wird kurz auf die Historie von Turing Tests eingegangen, 
bevor die verschiedenen CAPTCHA-Methoden und Alternativen grundlegend erläutert werden. 
Zuletzt wird ein grundlegendes Verständnis für UX geschaffen.

\section{Turing Tests}
\label{ch:basics:turing}
Alan M. Turing $($1912-1954$)$ ist einer der Mitbegründer der heutigen Informatik 
und legte mit seiner Forschung den Grundstein für die Entwicklung von künstlicher Intelligenz. 
In seinem Paper „On computable numbers, with an application to the Entscheidungsproblem“ \cite{turing} 
beschreibt er den Umgang mit „computable numbers“ und wie diese durch eine - später als Turing Machine bezeichnete - Maschine berechnet werden könnten. 
Hierbei kam er zur Erkenntnis, dass sich nicht alle mathematischen Probleme durch eine fixe Vorgehensweise lösen lassen. \cite[p.647]{geniusofturing} 

Aus dieser Erkenntnis heraus wurde der Turing Test entwickelt. 

\section{CAPTCHA}
%label
Bei der Abkürzung CAPTCHA handelt es sich um „completely automated public turing tests to tell computers and humans apart“. 
Sie werden genutzt, um Webseiten vor Angriffen durch Bots zu schützen. 

Fast identisch zu CAPTCHA sind sogenannte HIP – „Human Interactive Proofs“. 
Dieser Begriff hat sich entwickelt, da manche Tests nicht „public“ sind. \cite[p.1]{chellapilla} \cite{tutorial} 


\subsection{Textbasierte CAPTCHA}

Textbasierte CAPTCHA waren bereits 2008 die am häufigsten verwendete Art von CAPTCHA 


\subsection{Bildbasierte CAPTCHA}
\subsection{Audiobasierte CAPTCHA}
\subsection{Logik-CAPTCHA}
\subsection{Gamification}
\subsection{Alternativen zu CAPTCHA}
\subsubsection{Anti Spam Plugins}
\subsubsection{Multi-Faktor Authentifikation}
\subsubsection{Biometrie}
\subsubsection{Honeypots}

Eine Alternative zu klassischen CAPTCHA sind sogenannte Honeypots. 
Geprägt wurde der Begriff erstmals im Kalten Krieg als Spionagetechnik eingesetzt wurde. \cite[p.2]{joshi:2011} 

Auch heute werden Honeypots eingesetzt, und zwar in der IT-Sicherheit. 
Oft werden sie mit „Fallen“ assoziiert, welche Hacker anlocken sollen. 
Dadurch können Angriffsarten analysiert werden und „echte“ Systeme werden nicht angegriffen.

Doch auch im Kontext der Unterscheidung von Menschen und Maschine gibt es Honeypots. 
So kann man HTML-Inputfelder durch CSS verstecken, sodass diese nur durch Bots, welche den Quellcode der Seite scannen, ausgefüllt werden 
und nicht durch Menschen, da diese das Textfeld nicht sehen. 
Bei der Prüfung der Inputs kann nun überprüft werden, ob ein Bot in die Falle getappt ist. 
Nachzulesen sind solche Verfahren in verschiedenen Blogposts, wie \cite{perry:2019}.


\section{UX}

%Was ist ux, was hat das mit captcha zu tun (nutzungsfluss wird unterbrochen oder so)
%(Yan & Ahmad, 2008) 
