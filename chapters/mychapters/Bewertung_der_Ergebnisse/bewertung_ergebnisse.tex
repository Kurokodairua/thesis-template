\chapter{Bewertung der Ergebnisse}
\label{ch:bewertungergebnisse}

Die zuvor beschriebenen Bewertungen dienen lediglich als Orientierung, 
bei Benutzung der entwickelten Matrix sollten diese individuell neu durchgeführt werden,
da je nach Bedarf die vorgegebenen Leitfragen abgeändert oder anders interpretiert werden können.
Dadurch entsteht ein gewisser Grad an Subjektivität, welcher sich nicht vermeiden lässt.
Insbesondere bei der Bewertung der technischen Umsetzung muss individuell nachjustiert werden,
da die Gegebenheiten sehr individuell sein können und man beispielsweise von einer im Allgemeinen guten Dokumentation nicht auf Einzelfälle schließen kann.
Auch ist anzumerken, dass im Bereich Accessibility relativ streng bewertet wurde. 
Je nach Nutzergruppe und Anwendungsgebiet kann die Bewertung hier angepasst werden, sofern eine einfache Änderung der Gewichtung nicht ausreicht.

Aufgrund ihrer minimalinvasiven Natur sind CAPTCHAs ohne direkte Interaktion mit den Nutzer*innen immer zu bevorzugen.
Jedoch gibt es hier besonders bei reCAPTCHA vermehrt Sorgen zur Nutzung der gesammelten Daten durch Google. 
Außerdem müssen eventuell trotzdem noch klassische CAPTCHAs ausgefüllt werden, sofern die betrachteten Daten kein klares Ergebnis liefern konnten.