%*******************************************************
% Abstract
%*******************************************************
%\renewcommand{\abstractname}{Abstract}
\pdfbookmark[1]{Abstract}{Abstract}
\begingroup
\let\clearpage\relax
\let\cleardoublepage\relax
\let\cleardoublepage\relax

\begin{otherlanguage}{american}
	\chapter*{Abstract}
	Reducing the usage of resources is essential when maintaining a website. “Completely automated public turing tests to tell computers and humans apart”, in short CAPTCHA, are often used to prevent attacks from malicious bots. The completion of these tests is often complicated and disrupts the usage of the website. 
In the following paper we look at different techniques to tell humans and computers apart and how they influence the development and usage of websites and their security. For this purpose we design an evaluation matrix to rate different methods regarding their effort to complete, their accessibility, their technical feasibility and their security. Based on this evaluation we look at different fields of use to determine which CAPTCHA method is the best option overall.
\end{otherlanguage}

\newpage
\cleardoublepage

\begin{otherlanguage}{ngerman}
	\pdfbookmark[1]{Zusammenfassung}{Zusammenfassung}
	\chapter*{Zusammenfassung}
	Bei der Betreibung von Webseiten ist das Einsparen von Ressourcen essentiell. Häufig werden dazu „completely automated public turing tests to tell computers and humans apart“, im Sprachgebrauch meist als CAPTCHA abgekürzt, eingesetzt, um Angriffe durch Bots zu unterbinden. Das Ausfüllen dieser Tests ist oftmals kompliziert und stört bei der Benutzung. 
In der folgenden Arbeit wird betrachtet, wie sich verschiedene Techniken zur Unterscheidung von Menschen und Maschine auf die Erfahrung bei der Entwicklung und Nutzung von Webseiten sowie deren Schutz auswirken. Hierzu wird eine Bewertungsmatrix entwickelt. Mithilfe dieser Matrix werden ausgewählte Methoden hinsichtlich des Aufwands bei dem Ausfüllen des Tests, Accessability, technischer Umsetzbarkeit und Sicherheit beurteilt.  
Anschließend werden abhängig der Anforderungen verschiedener Einsatzgebiete Empfehlungen ausgesprochen, welche der Optionen die beste Gesamtleistung bieten.
\end{otherlanguage}

\endgroup

\vfill
