%*******************************************************
% Abstract in German
%*******************************************************
\begin{otherlanguage}{ngerman}
	\pdfbookmark[0]{Zusammenfassung}{Zusammenfassung}
	\chapter*{Zusammenfassung}
	Bei der Betreibung von Webseiten ist das Einsparen von Ressourcen essentiell. 
	Häufig werden dazu „completely automated public turing tests to tell computers and humans apart“, im Sprachgebrauch meist als CAPTCHA abgekürzt, eingesetzt, um Angriffe durch Bots zu unterbinden. 
	Das Ausfüllen dieser Tests ist oftmals kompliziert und stört bei der Benutzung.

	In der folgenden Arbeit wird betrachtet, wie sich verschiedene Techniken zur Unterscheidung von Menschen und Maschine auf die Erfahrung bei der Entwicklung und Nutzung von Webseiten sowie deren Schutz auswirken. 
	Hierzu wird eine Bewertungsmatrix entwickelt. 

	Mithilfe dieser Matrix werden ausgewählte Methoden hinsichtlich der Bedienfreundlichkeit bei dem Ausfüllen des Tests, Accessability, technischer Umsetzbarkeit und Sicherheit beurteilt. 
	 
	Anschließend werden abhängig der Anforderungen verschiedener Einsatzgebiete Empfehlungen ausgesprochen, welche der Optionen die beste Gesamtleistung bieten.

\end{otherlanguage}
